\section*{\scshape\textbf{Dask}}

  A diferencia de \ti{DuckDB}, con \ti{Dask} se tiene que cargar las tablas
  al contexto de ejecución. Por lo cual, luego de cargar las tablas del formato
  \ti{.parquet} se ejecutó el siguiente código:

  \begin{listing}[!ht]
    \inputminted[
      breaklines,
      fontfamily=SourceSansPro-TLF,
      firstline=41,
      lastline=49
    ]{python}{../src/LAB_07.py}
    \caption{Dask SQL python files}
    \label{dask:tablas}
  \end{listing}
  
  Donde cada query del $1$ al $23$ se ejecutaron $5$ veces para obtener un
  promedio de los tiempos de ejecución, los tiempos de ejecución se pueden
  encontrar en el archivo \ti{data/dask\_times.csv}. Por otro lado, para
  la ejecución de cada consulta se procedió de la siguiente manera:

  \begin{listing}[!ht]
    \inputminted[
      breaklines,
      fontfamily=SourceSansPro-TLF,
      firstline=15,
      lastline=29
    ]{python}{../src/LAB_07.py}
    \caption{Dask SQL python files}
    \label{master_service}
  \end{listing}

  Como se puede ver en el código \ti{python} se leen las tablas de los archivos
  \ti{.parquet}, asi como también se leen las consultas. Para finalmente hacer
  las \ti{queries} respectivas con el commando \ti{sql} de \ti{dask}; sin embargo, debido
  a la naturaleza de \ti{Dask} la consulta será ejecutada al llamar al commando\ti{compute}.
  Cada uno de los tiempos de ejecución se guardan en el archivo \ti{dask\_times.csv} como también
  los resultados de la query en su respectivo archivo \ti{.csv}, todo esto para
  hacer las comparaciones respectivas en los siguientes experimentos.
